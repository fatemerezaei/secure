
\section{Related Work}\label{related}
\red{i spotted some formatting/ grammar issues. please proofread.}

\red{shrink related work to make room for important text you removed. this is too long. }

In this section, we overview previous work on 
classifying different protocols and discuss previous attacks on \bc cryptocurrency.

\subsection{Protocol Classification}
There is extensive work in literature trying to classify different applications or protocols in the network. Previously, researchers were focused on classification according to the port numbers~\cite{tcp_p2p,ports,port_mad,payload_p2p} and payload~\cite{payload_p2p,payload_content,payload_app,payload_moore}. Since many applications use uncertain port numbers~\cite{ports}, or some encrypt their payloads, researchers adopted new methods for protocol classification.
Recent techniques use statistics such as packet sizes and timings for classification~\cite{real_enc,web_p2p,blinc,prot_fing,semi,trafficClassSVM,svm2}. 
\begin{comment}
For example, in~\cite{blinc}, authors use a three-level classification mechanism -social, functional, and application level- to classify web, p2p, chat, media, gaming, data transfer, streaming, and network management traffic. Their result shows that their method can classify $80$ $-$ $90\%$ of traffic flows with more than $95\%$ accuracy. In~\cite{prot_fing}, authors use packet sizes, inter-arrival times, and their orderings to distinguish between HTTP, SMTP, SSH, etc. They use \textit{anomaly score} to check if a flow is statistically compatible with one of the traffic classes. Their approach results in more than $90\%$ hit-ratio for all the considered classes and false positive around $6\%$ in the worst scenario.

In~\cite{web_p2p}, authors use duration, number of bytes, and number
of packets to classify flows at the network core where only one side of the flow is available (server-to-client or client-to-server). They use the K-means algorithm for clustering traffics, including Web, P2P, FTP, and use byte and flow accuracy to evaluate their mechanism. 
They show that server-to-client datasets give the best performance: 95$\%$ and 79$\%$ flow and byte accuracy, respectively.\end{comment}

In some of the studies, researchers apply machine learning techniques on these statistics to classify different applications.
For example, in~\cite{semi}, authors take a semi-supervised approach to classify a variety of applications such as FTP, HTTP, P2P. % They show that their technique could reach high accuracy by using a few labeled data and a large number of unlabeled samples. 
In~\cite{trafficClassSVM,svm2}, authors use SVM classifier to distinguish different applications such as WWW, Mail, and FTP. Also, in~\cite{svm2}, authors use SVM to classify a broad application category such as mail, buck traffic, service. %They show that using some statistics such as packet length, byte ratio of sent and received packets, etc. their method reaches more than $95\%$ accuracy.
In~\cite{MooreZ05}, Andrew W. Moore et al. use two refinements on the Bayesian technique and show that they can reach $95\%$ of accuracy on Internet traffic classification. Moreover, in~\cite{bays2}, authors use the Bayesian neural network to classify Internet traffic using just header-derived statistics. %They show that their method is comparable to the ones that use payload or other sophisticated traffic processing techniques.
 In ~\cite{mlEffi}, Wei Li et al. use C4.5~\cite{tree} decision trees to classify Internet traffic such as mail, gaming, database, and browsing. They reach an accuracy of $99.8$, using $12$ features collected at the start of the flow.
  In~\cite{survey_ml,myth}, authors survey the papers on Internet traffic classification using machine learning techniques. 
  %To the best of our knowledge, we are the first ones who attempt to distinguish \bc traffic from other applications. We design several binary classifiers to detect \bc traffic in the presence of small background noises. Furthermore, we utilize neural networks to detect \bc traffic in the presence of more complex background noises such as browsing more than one website (up to $5$) or running applications from CAIDA (up to $5$ number of different applications).
 

  \subsection{Attacks on \bc Cryptocurrency}
 In this section, we discuss the previous attacks on the \bc network. 
 In~\cite{hijack}, the authors discuss routing attacks, their impact, and possible countermeasures. They study partitioning and delay attacks by investigating node-level and network-wide attacks for both. %They adopt delay attack to slow down the propagation of blocks toward a set of nodes. Moreover, they use partitioning attack to isolate a set of nodes in the \bc network by diverting or cutting all the connections from those nodes to the network.
 
In~\cite{eclipse}, authors design eclipse attack in which the attacker isolate a victim from its peers by monopolizing its all incoming and outgoing connections, and filters her view of the network. Doing so, the attacker wastes her computing power on an outdated view of the network.

 %Doing so, the attacker can filter the victim's view of the network and therefore, force the victim to waste her computing power on an outdated view of the network. Furthermore, there have been many deanonymization attacks on \bc. 
In~\cite{analysis_anon,privacy_anon,struct,fist,quant}, authors use \bc transaction patterns
to link users (or link transactions) using some side information.% For example, in \cite{fist}, the authors design some heuristics to utilize the transaction graph to link transactions to the addresses and eventually infer the identities linked to them.

In~\cite{deanom1}, the authors propose a technique to link the public key of a user to her address or link her transactions. They show that they could launch their attacks even when the clients are behind NAT using only a few number of machines.
 
 In~\cite{double}, they analyze the security of using \bc for fast payments and show that the current \bc system is not secure unless they integrate \bc network with some detection mechanism.
 Also, they study double spending attacks on fast payments and  implement a method to prevent it.
 In~\cite{majority}, the authors introduce selfish-mine in which a pool can obtain a revenue larger than its share of mining power by forcing the honest miners to work on the wrong block, and therefore, waste their computing power.
 
 %. Selfish miners hide their mined blocks to create a private branch and selectively reveal them. Doing so, they force the honest miners to move to the new head of the blocks and waste their computing power on the wrong block.
 
 
In~\cite{refund}, the authors propose a new attack on the \bc payment system that exploits some authentication vulnerability, or some weakness on the refund procedure. They suggest a revision on the \bc payment protocol, which prevents both attacks.
%\fatemeh{this is our attack:}

%In our paper, we tackle the privacy of \bc users by detecting their \bc traffic even when
%a user tries to hide its \bc traffic using cover traffic or tunneling through Tor.

 %There are many reasons that a user needs to hide its \bc traffic. One of the main reasons is that in some countries having \bc is illegal.


% In ~\cite{traffic_bitcoin}
 %In ~\cite{selfish},